%\documentclass[12pt,letterpaper,doublespace, oneside]{article}
\documentclass[17pt]{extarticle}  % Use 14pt globally




%Here are the various packages I use. Some may be duplicated. 
\usepackage{enumerate}
\usepackage{etoolbox}
\usepackage{amsmath,amsthm,amssymb} %this is THE math package
\usepackage{mathtools} %to beef up the above package, more math!
\usepackage{tikz} %for drawing 
\usepackage{graphicx} %for including graphics
\usepackage{fancybox} %for some nice formatting options
\usepackage{hyperref} %for referencing
%hidelinks removes red and green boxes
\usepackage{varwidth} %for some nice width control
\usepackage{mdframed} %for framed environments
\usepackage{mathrsfs} %more math fonts
\usepackage{xcolor} %THE colour package
\usepackage{setspace}
\usepackage{multirow,array}
\usepackage{caption}
\usepackage[utf8]{inputenc}
\usepackage{pdfpages}
\usepackage[numbers, square]{natbib}
\usepackage{titlecaps}
%\usepackage[paper=a3paper]{geometry}
\usepackage{tabularx}
\usepackage{cleveref}
%attempting to capitalize citations
\usepackage [english]{babel}
\usepackage [autostyle, english = american]{csquotes}
\usepackage{xstring}
\usepackage{nameref}
\usepackage{amsthm}
\usepackage{lipsum}
\usepackage{enumitem}
\usepackage{titlesec}
\usepackage{float}
\usepackage[normalem]{ulem}
\usepackage{booktabs} % Include in preamble
%Here are the various packages I use. Some may be duplicated. 






%Not sure what these do, but they get the job done%
%\usepackage[notes,backend=biber]{biblatex-chicago}
%\usepackage[authordate-trad,backend=biber]{biblatex-chicago}
\MakeOuterQuote{"}
%some colours
\definecolor{firebrick}{RGB}{178,34,34}
\definecolor{teal}{RGB}{0,128,128}
\definecolor{indigo}{RGB}{75,0,130}
\definecolor{darkblue}{rgb}{0.0,0.0,.7}
\definecolor{darkred}{rgb}{0.6,0.0,0.0}
\definecolor{lightgrey}{RGB}{220, 220, 220}
\definecolor{darkgrey}{HTML}{878787}
\definecolor{forest}{HTML}{004a2f}
\definecolor{dirt}{HTML}{5d4728}
\definecolor{newblue}{HTML}{004fd9}
\definecolor{paleyellow}{HTML}{FFFFD3}
\renewcommand{\thesection}{}  % Remove numbering from \section
\renewcommand{\thesubsection}{}  % Remove numbering from \subsection
\renewcommand{\thesubsubsection}{} 
\DeclareMathAlphabet{\mathbx}{U}{BOONDOX-ds}{m}{n}
\DeclareMathOperator*{\E}{\mathbb{E}}
\SetMathAlphabet{\mathbx}{bold}{U}{BOONDOX-ds}{b}{n}
\DeclareMathAlphabet{\mathbbx} {U}{BOONDOX-ds}{b}{n}
\doublespacing
%\usepackageA{hyphenat}
\DeclareCaptionLabelFormat{blank}{}
\let\cleardoublepage\relax
%Not sure what these do, but they get the job done%








\begin{document}

\begin{titlepage}
    \centering
    \vspace*{\fill}

    \textsc{\Huge The Causal Impact of anti-Depressants at the Individual Level}\\[2em]

	\textsc{\Large Noah Dixon}\\[2em]
	
    %{\Large Noah Dixon}\\[3em]

	\textbf{\textsc{\LARGE {\color{darkred}9-4-24} }}
	
	\vspace*{\fill}

\end{titlepage}

%\title
%\name
%\data
%\maketitle
%\thispagestyle{empty}
\newpage
\UseRawInputEncoding

%defining Chicago as purely capitalized
\newcommand{\capitalizeTitle}[1]{%
    \StrSubstitute{#1}{ }{~}[\title]%
    \expandafter\capitalizetitle\expandafter{\title}%
}

\newcommand{\capitalizetitle}[1]{%
    \expandafter\StrSubstitute\expandafter{#1}{~}{ }[\Title]%
    \expandafter\StrSubstitute\expandafter{\Title}{ a }{ A }[\Title]%
    \expandafter\StrSubstitute\expandafter{\Title}{ an }{ An }[\Title]%
    \expandafter\StrSubstitute\expandafter{\Title}{ and }{ And }[\Title]%
    \expandafter\StrSubstitute\expandafter{\Title}{ as }{ As }[\Title]%
    \expandafter\StrSubstitute\expandafter{\Title}{ at }{ At }[\Title]%
    \expandafter\StrSubstitute\expandafter{\Title}{ but }{ But }[\Title]%
    \expandafter\StrSubstitute\expandafter{\Title}{ by }{ By }[\Title]%
    \expandafter\StrSubstitute\expandafter{\Title}{ for }{ For }[\Title]%
    \expandafter\StrSubstitute\expandafter{\Title}{ from }{ From }[\Title]%
    \expandafter\StrSubstitute\expandafter{\Title}{ in }{ In }[\Title]%
    \expandafter\StrSubstitute\expandafter{\Title}{ into }{ Into }[\Title]%
    \expandafter\StrSubstitute\expandafter{\Title}{ near }{ Near }[\Title]%
    \expandafter\StrSubstitute\expandafter{\Title}{ of }{ Of }[\Title]%
    \expandafter\StrSubstitute\expandafter{\Title}{ on }{ On }[\Title]%
    \expandafter\StrSubstitute\expandafter{\Title}{ onto }{ Onto }[\Title]%
    \expandafter\StrSubstitute\expandafter{\Title}{ or }{ Or }[\Title]%
    \expandafter\StrSubstitute\expandafter{\Title}{ the }{ The }[\Title]%
    \expandafter\StrSubstitute\expandafter{\Title}{ to }{ To }[\Title]%
    \expandafter\StrSubstitute\expandafter{\Title}{ under }{ Under }[\Title]%
    \expandafter\StrSubstitute\expandafter{\Title}{ upon }{ Upon }[\Title]%
    \expandafter\StrSubstitute\expandafter{\Title}{ with }{ With }[\Title]%
    \expandafter\StrSubstitute\expandafter{\Title}{ within }{ Within }[\Title]%
    \expandafter\StrSubstitute\expandafter{\Title}{ without }{ Without }[\Title]%
    \expandafter\StrSubstitute\expandafter{\Title}{ and }{ And }[\Title]%
    \expandafter\MakeUppercase\expandafter{\Title}%
}

\newcommand{\zz}{\mathbx Z}   %blackboard bold Z
\newcommand{\qq}{\mathbx Q}   %blackboard bold Q
\newcommand{\ff}{\mathbx F}   %blackboard bold F
\newcommand{\rr}{\mathbx R}   %blackboard bold R
\newcommand{\nn}{\mathbx N}   %blackboard bold N
\newcommand{\cc}{\mathbx C}   %blackboard bold C
\newcommand{\dd}{\mathsf D}   
\newcommand{\id}{\operatorname{id}} %for identity map
\newcommand{\im}{\operatorname{im}} %for image of a function
\newcommand{\dom}{\operatorname{dom}} %for domain of a function
\newcommand{\abs}[1]{\left\lvert#1\right\rvert} %for absolute value
\newcommand{\norm}[1]{\left\lVert#1\right\rVert} %for norm
\newcommand{\modar}[1]{\operatorname{mod}{#1}} %for modular arithmetic
\newcommand{\set}[1]{\left\{#1\right\}} %for set
\newcommand{\setp}[2]{\left\{#1\ :\ #2\right\}} %for set with a property
\newcommand{\lag}{\mathcal{L}}

\renewcommand\thepage{}

%Re-defined notations
\renewcommand{\epsilon}{\varepsilon}
\renewcommand{\phi}{\varphi}
\renewcommand{\emptyset}{\varnothing}
\renewcommand{\geq}{\geqslant}
\renewcommand{\leq}{\leqslant}
\renewcommand{\Re}{\operatorname{Re}}
\renewcommand{\Im}{\operatorname{Im}}

%----------------------------------------------
%Theorem, Lemma, Example, Definition etc. environments

%By default, the text in these environments are italicised
\theoremstyle{theorem}
\newtheorem{theorem}{Theorem}
\theoremstyle{proposition}
\newtheorem{proposition}{Proposition}
\theoremstyle{definition}
\newtheorem{definition}{Definition}
%\newtheorem{theorem}{Theorem}
\theoremstyle{lemma}
\newtheorem{lemma}[theorem]{Lemma}
\theoremstyle{corollary}
\newtheorem{corollary}[theorem]{Corollary}
%\newtheorem{proposition}[theorem]{Proposition}
%\theoremstyle{definition} %makes text non-italicized
\theoremstyle{example}
\newtheorem{example}[theorem]{Example}
\theoremstyle{remark}
\newtheorem{remark}[theorem]{Remark}
\theoremstyle{conclusion}
\newtheorem{conclusion}[theorem]{Conclusion}

\tableofcontents
\pagebreak

\section{}
\begin{proof}

\end{proof}



%\newpage
%\bibliographystyle{chicago}
%\bibliography{BEERthesisREAL.bib}
%\addcontentsline{toc}{section}{References}
\end{document}

